\documentclass{book}

\usepackage[english]{babel}
\usepackage{listings}

\newcommand\ensembl{EnsEMBL }

\title{Common Lisp Computational Biologiy - Handbook}

\begin{document}

\titlepage

\chapter{Introduction}


\section{Motivation for CLCB}

Lisp is fun. And it was felt that for a particular problem at hand, 
there was little gained by using Bio* libraries for Java, Python or Perl.
Hence, the decision was made to implement in Lisp and from day one kept in
mind to offer the fun to others, too.

One can program Lisp as if it was Perl, which is referred to as
''declarative''. What Lisp once was special for is the {\em fun}ctional
programming. Everyone has experienced what functional programming is alike
who has specified the first argument to the {\em sort} routine in Perl,
which is the expression deciding if the objects \$a or \$b is larger, with
a non-trivial expression.  In Lisp,
such functions that do not need a name are called $\lambda$ expressions.
One takes such, optionally gives them a name to make them real functions,
and applies them to lists (or another structure) of objects. To the biologically minded in us,
such functions are much like an enzyme that is acting on some sequence of
nucleotides. In other languages one only ''calls'' functions by name. In Lisp, functions are something tangible. Admittedly, these other languages do offer constructs that allow for a functional
programming to some degree. But with Lisp it is the other way around.

But exactly {\em why} is it so much fun? There are probably many answers
to this questions.  For one, the language is very clean, even for concepts
that came into fashion twenty years after Lisp's conception. Take
for instance the overloading of methods when introducing a subclass.
When interating over a set of zoo animals, there is no need for the
typical switch statement to learn about the object's real type and then a
cast to call the real method. If in C or Java you would do something alike

\lstset{language=C}
\begin{lstlisting}
	for(animal *a= alist->first(); a->next();) {
		if ("dog"==typeof(a)) {
			dog *d = (dog *)a;
			d->speak();
		}
		else if ("cat"==typeof(a)) {
		...
		}
	}
\end{lstlisting}

to hear the miows and barks.  In Lisp, you'd get the sound directly for
any animal, also for the ones not originally anticipated in that zoo loop.
For bioinformatics, with its zoo of sequence types and their subtypes,
the genes and what was formerly known as "junk", such cleanliness is essential.

There are other reasons $\ldots$

\section{How to learn Lisp}

The Net is full of tutorials that introduce to the language
Lisp. We found the instructions at the GMU very good to follow
(http://www.cs.gmu.edu/~sean/lisp/LispTutorial.html). A much recommended
text book is

	Peter Seibel\\
	Practical Common Lisp\\
	ISBN 1-59059-239-5\\
	Apress

A more biologically inclined reader of this document may decide
to leave the syntax of the language aside while first passing through
this document and instead concentrate more on the understanding
of the classes that represent biological entities in CLCB.

\section{Instant access to CLCB without reading this documentation}

CLCB is developed with SBCL but should be compatible with other
implementations with Lisp. This section gives you a head start into the
code. Frankly, the real documentation is in the code, literally, this is
where '':documentation'' slots allow for the annotation of what was
programmed and we prepare to parse that information from the code into
this document. But this is only now being worked on.
Unless you have a colleague who is already proficient in CLCB, 
there is no alternative to jumping right in.

\subsection{Setting up the interpreter}

In your UNIX shell change the directory to where the README 
of CLCB is. The UNIX command ''pwd'' informs about the current
working directory. Then start the Lisp interpreter, possibly just
by typing ''sbcl''. The interpreted now needs to learn about 
where to find your CLCB source code. To achieve that, in the lines below
substitute ''path/to/CLCB'' with that the command ''pwd'' just sent to
the screen and type it into the interpreter. You'd need to do this
every time you start up. Your interpreter will have files it reads at
startup and you should in the longer run move these lines there.

\lstset{language=lisp}
\begin{lstlisting}
(pushnew <path/to/CLCB> asdf:*central-registry*)
(pushnew <path/to/CLCB/ensembl> asdf:*central-registry*)
(asdf:oos 'asdf:load-op :clcb) or
(asdf:oos 'asdf:load-op :clcb-ensembl)
(in-package :ensembl)
\end{lstlisting}

\subsection{First hands on experience}

When looking at the examples given below you need to keep in mind that
computer scientists try hard to distinguish between the data and the
way it is presented to the user. The reason is that the same data
is interpreted differently in different contexts and there is hence
no way to know for sure how the data shall be shown to meet the user's
expectations. The examples below give references to the data, indicating
that all data is available to the programmer. This is great to the
programmer, saving lots of work. An abstract way to present the data,
i.e, as LaTeX tables, ASCII with white space or tab delimited, is still
being designed.

The code below retrieves a gene from \ensembl, retrieves its transcripts and finally the first transcript's protein protein product. Be aware that the translation process uses ''*'' to refer to the result of the previous command.

\begin{lstlisting}
ENSEMBL> (fetch-by-stable-id ``ENSG00000000005'')
=> #<GENE {C974EC1}>

(defparameter *gene* (fetch-by-stable-id ``ENSG00000000005''))

ENSEMBL> (transcripts *gene*)
=> (#<TRANSCRIPT {CAB4751}>)

ENSEMBL> (translation (car *))
=> #<TRANSLATION {CAF4C49}>
\end{lstlisting}

Retrieval of exons that contribute to the first transcript.
\begin{lstlisting}
ENSEMBL> (exons (first (transcripts *gene*)))
=> 
(#<EXON {CB4F9E9}> #<EXON {CB5C809}> #<EXON {CB69539}> #<EXON {CB76249}>
 #<EXON {CB831E9}> #<EXON {CB8FF09}> #<EXON {CB9CC09}> #<EXON {CBA98E9}>
 #<EXON {CBB6619}> #<EXON {CBC3329}> #<EXON {CBD0011}> #<EXON {CBDCD09}>
 #<EXON {CBE9A09}> #<EXON {CBF6729}> #<EXON {CC03421}> #<EXON {CC10149}>
 #<EXON {CC1CE61}> #<EXON {CC2BB61}> #<EXON {CC3A8A9}> #<EXON {CC47AD1}>
 #<EXON {CC56809}>)
\end{lstlisting}

Retrieval of features of the first transcript's protein product.

\begin{lstlisting}
ENSEMBL> (protein-features (translation (car (transcripts *gene*))))
(#<PROTEIN-FEATURE {CD15729}> #<PROTEIN-FEATURE {CD178C9}>
 #<PROTEIN-FEATURE {CD18591}> #<PROTEIN-FEATURE {CD19269}>
 #<PROTEIN-FEATURE {CD19F31}> #<PROTEIN-FEATURE {CD1ABF9}>
 #<PROTEIN-FEATURE {CD1B8C9}> #<PROTEIN-FEATURE {CD1C591}>)
ENSEMBL> (mapcar #'protein-feature-type
                 (protein-features (translation (car (transcripts *gene*)))))
("Pfam" "Superfamily" "Superfamily" "Smart" "low_complexity" "low_complexity"
 "low_complexity" "Prosite_profiles")
\end{lstlisting}

Genomic coordinates of the protein features. But where is the chromosome
and where the organism? Well, normally you know about these already and they
are invariant whenever one is inspecting a particular protein that is not
a product of a translocation $\ldots$ hm $\ldots$ . It is the gene that can
be queries for the chromosome.

\begin{lstlisting}
(mapcar #'dna-sequence-interval
                 (protein-features (translation (car (transcripts *gene*)))))
(#<DNA-SEQUENCE {D1DF909}>
 #<MULTI-INTERVAL (#<DNA-SEQUENCE {D1FD389}> #<DNA-SEQUENCE {D1FE011}>)>
 #<MULTI-INTERVAL (#<DNA-SEQUENCE {D21BEF1}> #<DNA-SEQUENCE {D21CB71}> #<DNA-SEQUENCE {D21D7F1}>)>
 #<DNA-SEQUENCE {D2D2A01}> #<DNA-SEQUENCE {D2EA241}> #<DNA-SEQUENCE {D3019E1}>
 #<MULTI-INTERVAL (#<DNA-SEQUENCE {D320179}> #<DNA-SEQUENCE {D320DF9}>)>
 #<DNA-SEQUENCE {D342429}>)
ENSEMBL> (mapcar #'(lambda (x) (genomic-coordinates (dna-sequence-interval x)))
                 (protein-features (translation (car (transcripts *gene*)))))
((204639 . 204885) (204630 . 205434) (250229 . 255355) (204639 . 204918)
 (242551 . 242602) (216822 . 216855) (204921 . 205473) (204639 . 204888))
ENSEMBL> (chromosome *gene*)
"18"

\end{lstlisting}

\chapter{Packages}

\section{\ensembl}

The \ensembl database provides insights into 
\begin{itemize}
\item genomic sequences,
\item genes,
\item transcripts,
\item proteins,
\item sequence variation and
\item intergenomic comparisons.
\end{itemize}
The initiative comes with a Perl-generated web site, an \ensembl Perl
API and was instrumental for the development of BioPerl. So, if they
are all using Perl, why Lisp? Well, \ensembl uses an old release of
BioPerl and has experimented with a Java interface \ensembl, which was
dumped again. For an easier access to the data on the database level,
the effort BioMart was initiated.

BioMart takes the core data of \ensembl and prepares views on the data that
facilitate dramatically the preparationof everyday's analyses. However,
for anything out of the row, it is not prepared. The \ensembl classes
presented here take a different approach. They follow the same logic of
BioMart and some of the \ensembl API classes to retrieve data from the core
tables. Other than BioMart, all interactions are dynamically determined
and new interactions are evolving just as they are needed. Other than
BioPerl, only minimal code is needed for that effort and the results
are immediately accessible to further analyses. Add to that all the
differences that Lisp makes by itself.

\subsection{Preparation of \ensembl classes}

The classes in the clcb-ensembl package basically represent the tables of
the \ensembl core MySQL tables. An instance of that class shall represent 
a single entry in the table of the same name. With possibly a few transient
exceptions, the unique identifiers of a row have a single attribute with the 
name which is a concatenation of the table name and the suffix 'id'. Other
tables refer to entries of the prior table with an attribute of the same name.
Consequently, the formulation of queries in Ensembl databases seems straight
forward.

Lisp has the flexibility to utilise that straight-forwardness with
macros. One passes them a table name, the attributes, and the Lisp macros
will figure out themselves what the class definition should look alike.
In a way, at least for the standard attributes, it would not have benn required to
specify the attributes manually since these could be read from the database dynamically.
But it is fine as it is already.

\subsubsection{Preparation of views on Ensembl}

The first macro automates the specification views on the Ensembl data. These
can be understood as resulting from an arbitrary SQL query and much like in 
SQL, the results get the character of a class themselves.

\begin{lstlisting}
(defmacro def-ensembl-view (name superclasses slots &rest options)
  "Define an EnsEMBL class."
  (let ((supers (remove-duplicates (append superclasses '(ensembl-object))))
        (cl-options
         (nconc (list :base-table (substitute #\_ #\- (string-downcase name)))
                options)))
    `(def-view-class ,name ,supers ,slots ,cl-options)))
\end{lstlisting}

Below you see the macro applied to specify an abstract view on DNA sequence data:
\begin{lstlisting}
(def-ensembl-view dna-sequence ()
  ((seq-region-id :type integer
                  :db-constraints :not-null
                  :initarg :seq-region-id)
   (seq-region-start :type integer
                     :db-constraints :not-null
                     :initarg :seq-region-start)
   (seq-region-end :type integer
                   :db-constraints :not-null
                   :initarg :seq-region-end)
   (seq-region-strand :type integer
                      :db-constraints :not-null
                      :initarg :seq-region-strand)
   (seq-region :accessor seq-region
               :db-kind :join
               :db-info (:join-class seq-region
                         :home-key seq-region-id
                         :foreign-key seq-region-id
                         :set nil))
   (previous-element-fn :accessor previous-element-fn
                        :initform #'1-
                        :db-kind :virtual
                        :allocation :class)
   (next-element-fn :accessor next-element-fn
                    :initform #'1+
                    :allocation :class
                    :db-kind :virtual)
   (element-member-fn :accessor element-member-fn
                      :initform
                      #'(lambda (set x)
                          (with-slots (seq-region-start seq-region-end) set
                            (and (integerp x)
                                 (<= seq-region-start x seq-region-end))))
                      :allocation :class
                      :db-kind :virtual)))
\end{lstlisting}

And for more 1:1 links between classes and tables, the def-ensembl-class is available.

\begin{lstlisting}
(defmacro def-ensembl-class (class-name slots &key dna-sequence stable-id-char)
  (with-symbol-prefixing-function (table-concat class-name)
    `(progn
       ,(when stable-id-char
              `(progn
                (def-ensembl-stable-id-view ,class-name)
                (def-fetch-by-stable-id-method ,class-name)
                (push (cons ,stable-id-char ',class-name) *char-class-alist*)))
       (def-ensembl-view ,class-name (,@(when dna-sequence '(dna-sequence)))
          ,(append ;; If the object has a stable id, we need to link
                   ;; to the appropriate table.
                   (when stable-id-char
                     `((stable-id
                        :db-kind :join
                        :db-info (:join-class ,(table-concat "-STABLE-ID")
                                  :home-key ,(table-concat "-ID")
                                  :foreign-key ,(table-concat "-ID")
                                  :set nil))))
                   ;; The slot definitions for the table.
                   slots)))))
\end{lstlisting}

The lines below show the representation of the exon class:
\begin{lstlisting}
(def-ensembl-class exon
  ((exon-id :type integer
            :db-type :key
            :db-constraints (:primary-key :not-null))
   (phase :type integer
          :accessor exon-phase)
   (end-phase :type integer
              :accessor exon-end-phase)
   (is-current :type integer)
   (exon-transcript :db-kind :join
                    :db-info (:join-class exon-transcript
                              :home-key exon-id
                              :foreign-key exon-id
                              :set nil)))
  :stable-id-char #\e
  :dna-sequence t)
\end{lstlisting}



\end{document}
